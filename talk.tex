\documentclass{beamer}
\usetheme{Montpellier}
\usecolortheme{orchid}
\definecolor{UniGray}{RGB}{93,93,93}
\setbeamercolor{title}{fg=UniGray}
\setbeamercolor{frametitle}{fg=UniGray}
\setbeamercolor{structure}{fg=UniGray}
\setbeamertemplate{bibliography item}[triangle]
\setbeamertemplate{navigation symbols}{} % remove navigation symbols

\usepackage[utf8]{inputenc}
\usepackage{amsmath}
\usepackage{listings}
\usepackage{graphics}
\usepackage{pgfpages}
\usepackage[backend=bibtex, hyperref=true]{biblatex}
\makeatletter
\def\blx@maxline{77}
\makeatother
\addbibresource{talk.bib}

\lstset{basicstyle=\fontsize{9}{13}\selectfont\ttfamily}

\setcounter{tocdepth}{1}

\newcommand{\transformation}[1]{\raisebox{0.3ex}{\tiny{--- #1 $\rightarrow$}}}
\newcommand{\bitransformation}[1]{\raisebox{0.3ex}{\tiny{$\leftarrow$ #1 $\rightarrow$}}}

\title{GnuPG}
\subtitle{Una introducción}
\author{Unai Zalakain\\
{\tiny \url{http://unaizalakain.info/talks/gpg.pdf}}\\
\includegraphics[height=.2\textheight]{qrlink}
}
\date{Jornadas de Tecnología y Cultura Libre,\\ Junio 2016}


\begin{document}

\frame{\titlepage}

\begin{frame}{Índice}
\tableofcontents
\end{frame}


\section{Qué es}
\begin{frame}{Qué es}
\begin{itemize}
    \item software criptográfico
    \item software libre
    \item permite cifrar [confidencialidad]
    \item permite firmar [autenticación, integridad, no repudio]
    \item permite crear redes de confianza
    \item interfaz de línea de comandos
    \item multitud de interfaces gráficas
    \item integración con clientes de correo
    \item integración con gestores de contraseñas
\end{itemize}
\end{frame}


\section{Tipos de cifrado}
\subsection{Cifrado simétrico}
\begin{frame}{Cifrado simétrico}
\begin{itemize}
    \item texto plano \bitransformation{clave} texto cifrado
    \item quien puede cifrar, puede descifrar
    \item todas las partes necesitan conocer la clave común
\end{itemize}
\end{frame}

\subsection{Cifrado asimétrico}
\begin{frame}{Cifrado asimétrico}
\begin{itemize}
    \item texto plano \transformation{clave pública destinatario} texto
        cifrado
    \item texto cifrado \transformation{clave privada destinatario} texto
        plano
    \item cualquiera que posea la clave pública puede cifrar
    \item solo quien posea la clave privada puede descifrar
    \item se puede cifrar para múltiples claves privadas
\end{itemize}
\end{frame}


\section{Firmas digitales}
\begin{frame}{Firmas digitales}
\begin{itemize}
    \item texto plano \transformation{clave privada emisor} texto firmado
    \item texto firmado \transformation{clave pública emisor} texto plano
    \item solo quien posea la clave privada puede firmar
    \item cualquiera que posea la clave pública puede comprobar
\end{itemize}
\end{frame}


\section{Redes de confianza}
\begin{frame}{Redes de confianza}
\begin{itemize}
    \item firma digital de una clave con otra
    \item certificación de que la clave firmada pertenece a la persona
        mencionada
    \item la confianza puede ser transitiva: red de confianza
    \item \emph{firmar solamente aquellas identidades de las que se esté
        absolutamente segura}
    \item confianza inversamente proporcional a la longitud de la cadena de
        firmas entre dos claves
    \item confianza proporcional al número de caminos sin escalafones comunes
        entre dos claves
    \item \emph{strong set}: la mayor red de confianza interconectada pública
\end{itemize}
\end{frame}


\subsection{Keysigning Parties}
\begin{frame}{Keysigning Parties}
\begin{itemize}
    \item antes:
        \begin{enumerate}
            \item enviar clave pública propia a la coordinadora
        \end{enumerate}
    \item durante:
        \begin{enumerate}
            \item verificar información sobre la clave pública propia
            \item verificar información sobre las claves públicas de las demás
            \item verificar identidad de las demás
        \end{enumerate}
    \item después:
        \begin{enumerate}
            \item firmar digitalmente las claves públicas verificadas
            \item enviar las claves públicas firmadas digitalmente a sus dueñas
            \item recibir la clave pública propia firmada digitalmente por las demás
            \item subir la clave pública propia firmada digitalmente por las
                demás a los servidores
        \end{enumerate}
\end{itemize}
\end{frame}


\section{Subclaves}
\begin{frame}{Subclaves}
\begin{itemize}
    \item clave maestra representa identidad
    \item firmas digitales, cifrado y autenticación delegadas a subclaves
    \item subclaves firmadas digitalmente por clave maestra
    \item clave maestra puede mantenerse separada y segura
    \item clave maestra solo necesaria para operaciones inusuales:
        \begin{itemize}
            \item modificación de claves
            \item firmas digitales de claves de otras
            \item generación de certificados de revocación
        \end{itemize}
    \item subclaves pueden ser revocadas sin afectar identidad principal
\end{itemize}
\end{frame}


\section{Taller práctico}


\subsection{Creando una clave}
\begin{frame}[fragile]{Creando una clave}
\begin{lstlisting}
gpg2 --full-gen-key
\end{lstlisting}
\begin{enumerate}
    \item seleccionar \texttt{RSA and RSA}
    \item seleccionar \texttt{4096} como tamaño
    \item fijar fecha de expiración
    \item proteger con una frase de paso fuerte
    \item certificado de revocación generado automáticamente
\end{enumerate}
\end{frame}


\subsection{Añadiendo una subclave para firmas digitales}
\begin{frame}[fragile]{Añadiendo una subclave para firmas digitales}
\begin{lstlisting}
gpg2 --edit-key sixto.rodriguez@riseup.net

addkey

save
\end{lstlisting}
\begin{enumerate}
    \item seleccionar \texttt{RSA (sign only)}
    \item seleccionar \texttt{4096} como tamaño
    \item fijar fecha de expiración
\end{enumerate}
\end{frame}


\subsection{Añadiendo una identidad}
\begin{frame}[fragile]{Añadiendo una identidad}
\begin{lstlisting}
gpg2 --edit-key sixto.rodriguez@riseup.net

adduid

save
\end{lstlisting}
\begin{enumerate}
    \item seleccionar \texttt{RSA (sign only)}
    \item seleccionar \texttt{4096} como tamaño
    \item fijar fecha de expiración
\end{enumerate}
\end{frame}


\subsection{Listando claves}
\begin{frame}[fragile]{Listando claves}
\begin{itemize}
    \item públicas
        \begin{lstlisting}
gpg2 -k
        \end{lstlisting}
    \item privadas
        \begin{lstlisting}
gpg2 -K
        \end{lstlisting}
\end{itemize}
\end{frame}

\subsection{Creando un certificado de revocación}
\begin{frame}[fragile]{Creando un certificado de revocación}
\begin{lstlisting}
gpg2 --gen-revoke sixto.rodriguez@riseup.net
\end{lstlisting}
\end{frame}


\subsection{Guardando una copia de seguridad}
\begin{frame}[fragile]{Guardando una copia de seguridad}
\begin{lstlisting}
gpg2 --export-secret-keys --armor > secret.gpg
gpg2 --export-keys --armor > public.gpg
\end{lstlisting}
\end{frame}


\subsection{Transfiriendo la clave maestra}
\begin{frame}[fragile]{Transfiriendo la clave maestra}
\begin{itemize}
    \item GPG2 guarda cada clave secreta en un fichero distinto
    \item pueden obtenerse los nombres de los fichero con:
        \begin{lstlisting}
gpg2 -K --with-keygrip
        \end{lstlisting}
    \item los ficheros se encuentran en
        \begin{lstlisting}
${GNUPGHOME}/private-keys-v1.d/<keygrip>.key
        \end{lstlisting}
    \item añadir/quitar fichero con clave secreta
\end{itemize}
\end{frame}


\subsection{Cifrando y descifrando}
\begin{frame}[fragile]{Cifrando y descifrando}
\begin{lstlisting}
echo "Ping" | \
gpg2 -e -a -r sixto@rodriguez@riseup.net | \
gpg2 -d
\end{lstlisting}
\end{frame}


\subsection{Firmando y comprobando}
\begin{frame}[fragile]{Firmando y comprobando}
\begin{lstlisting}
echo "Ping" | gpg2 -s -a | gpg2 -v
\end{lstlisting}
\end{frame}


\subsection{Servidores de claves}
\begin{frame}[fragile]{Servidores de claves}
\begin{lstlisting}
echo "keyserver hkps://hkps.pool.sks-keyservers.net" >> \
    ${GNUPGHOME}/gpg.conf
gpg2 --send-keys 0BAA856A317CD58B302BFAAD8A5015B1CCB5CBDE
gpg2 --search-key unai@gisa-elkartea.org
gpg2 --recv-keys 4B3F515D76ADA88FA8B5247BF1DB723A779DD1F7
\end{lstlisting}
\end{frame}


\subsection{Verificando una clave}
\begin{frame}[fragile]{Verificando una clave}
\begin{lstlisting}
gpg2 --sign-key unai@gisa-elkartea.org
gpg2 -a --export unai@gisa-elkartea.org > signed.key
# Enviar signed.key al correo mencionado en la clave
\end{lstlisting}
\end{frame}


\section{Consejos útiles}
\begin{frame}[allowframebreaks]{Consejos útiles}
\begin{itemize}
    \item firmar públicamente solamente aquellas claves a las que no nos importe
        estar públicamente relacionadas
    \item mantener la clave maestra privada separada y segura
    \item comunicarse con los servidores de claves sobre HKPS (HKP sobre TLS)
    \item comunicarse con un pool de servidores
    \item evitar descargar una clave de un servidor especificado por dicha clave
        \framebreak
    \item asegurarse del fingerprint de las claves descargadas
    \item generar certificados de revocación
    \item especificar fecha de expiración
    \item especificar preferencias de algoritmos de cifrado fuertes
    \item enviar claves firmadas por correo a su dueño en vez de subirlas a los
        servidores
\end{itemize}
\end{frame}


\section{Software de interés}
\begin{frame}[allowframebreaks]{Software de interés}
\begin{description}
    \item[gnupg2] más completo y con una mejor interfaz
    \item[gnupg-agent] gestor de claves privadas
    \item[gnupg-curl] soporte para HKPS
    \item[guncat] descifra texto parcialmente cifrado
    \item[pius] firma y envía por correo un grupo de claves
    \item[kgpg] GUI para GPG
        \framebreak
    \item[pass] gestor de contraseñas cifradas con GPG
    \item[qtpass] GUI para pass
    \item[PassFF] rellenado de formularios en Firefox a través de pass
        \framebreak
    \item[icedove] cliente de correo
    \item[enigmail] addon para cifrado con GPG para icedove
    \item[claws-mail] cliente de correo
    \item[claws-mail-smime-plugin] addon para cifrado con GPG para claws-mail
\end{description}
\end{frame}


\section{Keysigning Party}
\begin{frame}[fragile]{Keysigning Party}
Reforcemos la red de confianza!
\begin{enumerate}
    \item exportar la clave pública
        \begin{lstlisting}
gpg2 -a --export > public.key
        \end{lstlisting}
    \item subir la clave pública a \url{http://biglumber.com/x/web?keyring=891}
    \item acudir a la keysigning party
\end{enumerate}
\end{frame}


\begin{frame}[allowframebreaks]{Referencias}
    \nocite{*}
    \printbibliography
\end{frame}
\end{document}
