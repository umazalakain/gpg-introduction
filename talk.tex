\documentclass{beamer}
\usetheme{Montpellier}
\usecolortheme{orchid}
\definecolor{UniGray}{RGB}{93,93,93}
\setbeamercolor{title}{fg=UniGray}
\setbeamercolor{frametitle}{fg=UniGray}
\setbeamercolor{structure}{fg=UniGray}

\usepackage[utf8]{inputenc}
\usepackage{amsmath}

\let\olditem\item
\renewcommand{\item}{%
\olditem\vspace{3pt}}

\title{GnuPG}
\subtitle{Introducción}
\author{Unai Zalakain}
\date{Jornadas de Tecnología y Cultura Libre,\\ Junio 2016}


\begin{document}

\frame{\titlepage}

\begin{frame}
\frametitle{Índice}
\tableofcontents
\end{frame}


\section{Qué es}
\begin{frame}
\frametitle{Qué es}
\begin{itemize}
    \item software criptográfico
    \item software libre
    \item permite cifrar [confidencialidad]
    \item permite firmar [autenticación, integridad, no repudio]
    \item permite crear redes de confianza
    \item interfaz de línea de comandos
    \item multitud de interfaces gráficas
    \item integración con clientes de correo
    \item integración con gestores de contraseñas
\end{itemize}
\end{frame}


\section{Tipos de cifrado}
\subsection{Cifrado simétrico}
\begin{frame}
\frametitle{Cifrado simétrico}
\begin{itemize}
    \item texto plano {\tiny $\leftarrow$clave$\rightarrow$} texto cifrado
    \item quien puede cifrar, puede descifrar
    \item todas las partes necesitan conocer la clave común
\end{itemize}
\end{frame}

\subsection{Cifrado asimétrico}
\begin{frame}
\frametitle{Cifrado asimétrico}
\begin{itemize}
    \item texto plano {\tiny ---clave pública destinatario$\rightarrow$} texto
        cifrado
    \item texto cifrado {\tiny ---clave privada destinatario$\rightarrow$} texto
        plano
    \item cualquiera que posea la clave pública puede cifrar
    \item solo quien posea la clave privada puede descifrar
    \item se puede cifrar para múltiples claves privadas
\end{itemize}
\end{frame}


\section{Firmas digitales}
\begin{frame}
\frametitle{Firmas digitales}
\begin{itemize}
    \item texto plano {\tiny ---clave privada emisor$\rightarrow$} texto firmado
    \item texto firmado {\tiny ---clave pública emisor$\rightarrow$} texto plano
    \item solo quien posea la clave privada puede firmar
    \item cualquiera que posea la clave pública puede comprobar
\end{itemize}
\end{frame}


\section{Redes de confianza}
\begin{frame}
\frametitle{Redes de confianza}
\begin{itemize}
    \item firma digital de una clave con otra
    \item clave pública a autenticar {\tiny ---clave secreta
        confiante$\rightarrow$} clave pública firmada por confiante
    \item clave pública firmada por confiante {\tiny ---clave pública
        confiante$\rightarrow$} clave pública a autenticar
    \item solo quien posea la clave privada puede depositar confianza
    \item cualquiera que posea la clave pública puede comprobar confianza
    \item la confianza puede ser transitiva: red de confianza
    \item \emph{strong set}: la mayor red interconectada pública de confianza
\end{itemize}
\end{frame}


\section{Claves maestras y subclaves}
\begin{frame}
\frametitle{Claves maestras y subclaves}
\end{frame}

\iffalse
practico:

- creando una clave
- cifrar
- descifrar
- firmar
- comprobar firma
- confiar

- programas de interes
- consejos utiles


# Cifrando con GPG

~~~
$ echo "Ping" | gpg2 -a -e -r unai@gisa-elkartea.org
-----BEGIN PGP MESSAGE-----
Version: GnuPG v2

hQEMA72uLbJJfe3kAQf8CuOAetkklBslwARAMGy6k8fAqQlI9il0RhiQsMhr02NJ
4Rg+7DatRR1IX7rtcUDknc36J0sN+HuMQGEK94xzRnLr/yJaHKc12wHSEOu9vNsN
kqpGHrhhSiCQPIpCHd7ooJ5oCyuXnMOgbAWKD9GxtxR0ZEV62+/Rl+AUL0bNElLR
l+znmX5LWG5liZLftPMcjFkHqD1jBqZ9N5d5iOdL5daUDUFWqNqMpg1Mgu8XaB1k
8Z+1TUkUwQ66KKteerZJtFf0889nO7ZfxpuqDu8XL89D/JmTaxGQIMmq8EEc0/6K
sj9mPvG1WaSWoN9oYFHes1B9YNI5SNew/iSTgad9kNJAAU0H+qersTJbcTMNOhx2
HJrHopGHEjN8KGWJ1X5J+lxd2TM02oCitwkgJbl8qpYy6385MzuQIbSWoQh9iBYl
LA==
=B5zs
-----END PGP MESSAGE-----
~~~


---


# Descifrando con GPG

~~~
$ gpg2 -d < file_with_encrypted_message
gpg: encrypted with 2048-bit RSA key, ID 497DEDE4, created 2016-05-04
      "Unai Zalakain (GISA) <unai@gisa-elkartea.org>"
Ping
~~~
\fi

\end{document}
